\documentclass{beamer}

\usepackage[utf8]{inputenc}

\usepackage{graphicx} % support the \includegraphics command and options

\usepackage[parfill]{parskip} % Activate to begin paragraphs with an empty line rather than an indent

%%% PACKAGES
\usepackage{booktabs} % for much better looking tables
\usepackage{array} % for better arrays (eg matrices) in maths
\usepackage{paralist} % very flexible & customisable lists (eg. enumerate/itemize, etc.)
\usepackage{verbatim} % adds environment for commenting out blocks of text & for better verbatim
\usepackage{subfig} % make it possible to include more than one captioned figure/table in a single float
\usepackage{mathtools} % for the all important \coloneqq symbol
\usepackage{url} % for \url
\usepackage{IEEEtrantools} % for \IEEEeqnarray

%% Font things %%
\usepackage{amsmath}
\usepackage{amsfonts}
\usepackage{amssymb}
\usepackage{stmaryrd}
\usepackage{mathtools}
\usepackage{cmll} % Linear logic symbols!

%% Lists %%
\usepackage{enumerate}

%% Graphics %%
\usepackage{tikz}
\usetikzlibrary{cd}
\usetikzlibrary{patterns}
\usetikzlibrary{calc}

%% Changes to enumerate things so they look better %%

\makeatletter
\def\enumfix{%
\if@inlabel
 \noindent \par\nobreak\vskip-\topsep\hrule\@height\z@
\fi}

\let\olditemize\itemize
\def\itemize{\enumfix\olditemize}
\let\oldenumerate\enumerate
\def\enumerate{\enumfix\oldenumerate}

%% Random crap %%
\usepackage{xifthen}

\makeatletter
\def\thm@space@setup{%
  \thm@preskip=\parskip \thm@postskip=0pt
}
\makeatother

\makeatletter
\newcommand*{\relrelbarsep}{.386ex}
\newcommand*{\relrelbar}{%
  \mathrel{%
    \mathpalette\@relrelbar\relrelbarsep
  }%
}
\newcommand*{\@relrelbar}[2]{%
  \raise#2\hbox to 0pt{$\m@th#1\relbar$\hss}%
  \lower#2\hbox{$\m@th#1\relbar$}%
}
\providecommand*{\rightrightarrowsfill@}{%
  \arrowfill@\relrelbar\relrelbar\rightrightarrows
}
\providecommand*{\leftleftarrowsfill@}{%
  \arrowfill@\leftleftarrows\relrelbar\relrelbar
}
\providecommand*{\xrightrightarrows}[2][]{%
  \ext@arrow 0359\rightrightarrowsfill@{#1}{#2}%
}
\providecommand*{\xleftleftarrows}[2][]{%
  \ext@arrow 3095\leftleftarrowsfill@{#1}{#2}%
}
\makeatother

\newcommand{\catname}[1]{{\normalfont\textbf{#1}}}
\newcommand{\Rings}{\catname{CRing}}
\newcommand{\CAT}{\catname{CAT}}
\newcommand{\Top}{\catname{Top}}
\newcommand{\Set}{\catname{Set}}
\newcommand{\Cont}{\catname{Cont}}
\newcommand{\Sch}{\catname{Sch}}
\newcommand{\Mod}[1][]{\ifthenelse{\isempty{#1}}{\catname{Mod}}{#1\catname{mod}}}
\DeclareMathOperator{\sh}{Sh}
\newcommand{\Sh}[1][]{\ifthenelse{\isempty{#1}}{\sh}{\sh(#1)}}
\newcommand{\map}[3]{#2\xrightarrow{#1} #3}
\newcommand*\from{\colon}
\newcommand{\cmap}[3]{#1\from{}#2\to{}#3}
\newcommand\oppcat[1]{#1^{\mathrm{op}}}
\DeclareRobustCommand{\vmap}[3] {\begin{tikzcd} #2 \arrow[d, "#1"] \\ #3 \end{tikzcd}}
\newcommand{\partref}[1]{(\ref{#1})}
\newcommand{\intgrpd}[4] {#1 \xrightrightarrows[#3]{#4} #2}
\DeclareRobustCommand{\bigintgrpd}[4] {\begin{tikzcd}[ampersand replacement=\&] #1 \arrow[r, shift left=0.5ex, "#3"] \arrow[r, shift right=0.5ex, "#4"'] \& #2 \end{tikzcd}}

\usepackage{xspace}

\newcommand{\etale}{\'{e}tale\xspace}
\newcommand{\Etale}{\'{E}tale\xspace}

\def \inv {^{-1}}

\DeclareMathOperator{\id}{id}
\DeclareMathOperator{\op}{op}
\DeclareMathOperator{\pr}{pr}
\DeclareMathOperator{\pre}{{pre}}
\DeclareMathOperator{\et}{{\acute{e}t}}

\DeclareMathOperator{\Hom}{Hom}
\DeclareMathOperator{\Spec}{Spec}

\DeclareMathOperator{\ol}{ol}

\def\presuper#1#2%
  {\mathop{}%
   \mathopen{\vphantom{#2}}^{#1}%
   \kern-\scriptspace%
   #2}
\def\presub#1#2%
  {\mathop{}%
   \mathopen{\vphantom{#2}}_{#1}%
   \kern-\scriptspace%
   #2}

%% Our things %%

\newcommand{\neggame}[1]{\presuper{\perp}{#1}}
\newcommand{\tensor}{\otimes}
\newcommand{\sequoid}{\oslash}
\newcommand{\varsequoid}{\vartriangleleft}
\renewcommand{\implies}{\multimap}
\newcommand{\comp}[2]{#1 \circ #2}
\newcommand{\cprd}{\sqcup}
\newcommand{\G}{\mathcal G}
\newcommand{\C}{\mathcal C}
\newcommand{\F}{\mathcal F}
\newcommand{\suchthat}{\;\colon\;}
\newcommand{\varsuchthat}{\;\mid\;}
\newcommand{\OP}{\{O,P\}}
\renewcommand{\L}{\mathcal L}

\newcommand{\nsb}{N\sequoid\underline{\hspace{1.5ex}}\;}

%%% END Article customizations

\title{Games with ordinal sequence of moves}

\author{John Gowers and James Laird}

\begin{document}

\begin{frame}
  \titlepage
\end{frame}

\section*{Outline}
\begin{frame}
  \tableofcontents
\end{frame}

\section{Preliminaries}

\subsection{Our starting point - the Abramsky-Jagadeesan Games Model}

\begin{frame}
  \frametitle{Our starting point - the Abramsky-Jagadeesan Games Model}

  We start off with a few simple categories of games:

  \begin{itemize}
    \item Games with a winning condition:
      \[
        A=(M_A,\lambda_A,b_A,P_A,W_A)
        \]
      The winning condition prevents infinite internal chattering.
    \item Games without a winning condition:
      \[
        A=(M_A,\lambda_A,b_A,P_A)
        \]
      We have no winning condition, so we have to use \\
      partial strategies or require that our games be bounded.
    \item We have the tensor product ($\tensor$) and implication ($\implies$) \\
      given by interleaving games and a categorical semantics $\G$\\
      where a morphism from $A$ to $B$ is a strategy for $A\implies B$.
  \end{itemize}

\end{frame}

\subsection{Exponentials and the sequoid operator}

\begin{frame}
  \frametitle{Construction of an exponential for our game model}

  $N$ is a negative game ($b_N=O$).

  \[
    !N=(M_N\times\mathbb N,\lambda_N\circ\pr_1,O,P_{!N})
    \]

  $!N$ is made up of countably many copies of the game $N$,\\
  indexed by $\mathbb N$ and played in parallel, with the opponent\\
  switching games.  

  $!N$ has the structure of the cofree commutative comonoid\\
  generated by $N$:
  \[
    !N\implies !N\tensor !N
    \]

\end{frame}

\begin{frame}
  \frametitle{The sequoid operator on games}

  \begin{itemize}
    \item $N\sequoid L$ is a weakening of $N\tensor L$ in which the opponent must make the first move in the game $N$.
    \item It induces a monoidal category action of $\G$ on \\
      (a modified version of) itself:
      \[
        L \sequoid (M \tensor N) \cong (L \sequoid M) \sequoid N
        \]
      This is the basis of the definition of a \emph{sequoidal category}.
    \item Then the exponential $!N$ arises as the final coalgebra \\
      for the `sequoid on the left by $N$' functor $\nsb$
      \[
        !N\implies N\sequoid !N
        \]
  \end{itemize}

\end{frame}

\subsection{Coalgebras and the final sequence}

\begin{frame}[fragile]
  \frametitle{The final sequence}
  Let $\C$ be a category and $\cmap{\F}{\C}{\C}$ be an endofunctor.
  \[
    \begin{tikzcd}
      1
        & \F(1) \arrow[l]
          & \F^2(1) \arrow[l]
            & \F^3(1) \arrow[l]
              & \cdots \arrow[l] \\
      \F^\omega(1) \arrow[u] \arrow[ur] \arrow[urr] \arrow [urrr] \arrow[urrrr, phantom, xshift=5.0ex, yshift=-0.5ex, "\cdots"]
        & \F^{\omega+1}(1) \arrow[l]
          & \F^{\omega+2}(1) \arrow[l]
            & \F^{\omega+3}(1) \arrow[l]
              & \cdots \arrow[l] \\
      \F^{\omega2}(1) \arrow[u] \arrow[ur] \arrow[urr] \arrow [urrr] \arrow[urrrr, phantom, xshift=5.0ex, yshift=-0.5ex, "\cdots"]
        & \cdots \arrow[l]
          &
            & \\
      \arrow[u, phantom, "\vdots"]
    \end{tikzcd}
    \]

  If a final coalgebra for $\F$ exists, then it occurs as $\F^\alpha(1)$ \\
  for some $\alpha$ and the sequence stabilizes thereafter.
\end{frame}

\section{Stabilization ordinals for categories of games}

\begin{frame}
  \frametitle{Stabilization ordinals for sequoidal categories of games}

  \begin{itemize}
    \item If $N$ is an object in the category $\G$ of games without \\
      a winning condition, then the final sequence for $\nsb$ \\
      stabilizes by $\omega$.
      \pause
    \item If $N$ is an object in the category $\mathcal W$ of games with \\
      a winning condition, then the final sequence for $\nsb$ \\
      stabilizes by $\omega2$.
      \pause
    \item If $\C$ is a sequoidal category, do the final sequences \\
      corresponding to `sequoid on the left' functors \\
      always stabilize by $\omega2$?  (No.)
      \pause
    \item Can we replace $\omega2$ by some other ordinal to give us a recipe \\
      for constructing the exponential in any sequoidal category?
  \end{itemize}
\end{frame}

\section{Games with ordinal sequences of moves}

\subsection{What we end up with}

\begin{frame}
  \frametitle{Idea of transfinite games}

  \begin{itemize}
    \item For games without a winning condition, we may have a move \\
      at time $n$ for every $n\in\omega$.  
    \item For games with a winning condition, we have things happening \\
      at time $\omega$ as well.  
    \item We now take things further, and construct a category $\G(\alpha)$ \\
      for every ordinal $\alpha$.  
    \item We will get $\G\cong\G(\omega)$ and $\mathcal W\cong\G(\omega+1)$.
    \item There is a connection between the ordinal $\alpha$ and the \\
      stabilization ordinals for functors of the form
      \[
        \cmap{\nsb}{\G(\alpha)}{\G(\alpha)}
        \]
  \end{itemize}
\end{frame}

\subsection{Brief outline of the construction}

\begin{frame}
  \frametitle{A brief outline of the construction of the category of games played over the ordinal $\alpha$}

  Fix an ordinal $\alpha$.  We construct a sequoidal category of \\
  games of length $\alpha$ by mimicking the Abramsky-Jagadeesan \\
  construction:
  \[
    N=(M_N, \lambda_N, \zeta_N, P_N)
    \]

  \begin{itemize}
    \item $M_N$ is a set of moves.
    \item $\cmap{\lambda_N}{M_N}{\OP}$ is a function saying which player each \\
      move belongs to.
    \item $P_N$ is a prefix closed set of ordinal length sequences (plays) \\
      that take values in $M_N$ and are indexed by ordinals $<\alpha$
    \item $\cmap{\zeta_N}{P_N}{\OP}$ assigns a player to each play.
  \end{itemize}

\end{frame}

\subsection{Stabilization ordinals}

\begin{frame}
  \frametitle{The stabilization ordinal of the category $\G(\alpha)$}

  \begin{itemize}
    \item The behaviour of the stabilization ordinal for functors \\
      of the form $\nsb$ in the category $\G(\alpha)$ is complicated \\
      in general.
    \item However, the final sequence for such functors never stabilizes \\
      before $\alpha$.
    \item Therefore, we have constructed sequoidal categories of games \\
      with arbitrarily high stabilization ordinals: there is no \\
      general recipe for constructing this exponential using only \\
      the application of the functor and passing to small limits.
  \end{itemize}
\end{frame}

\end{document}
