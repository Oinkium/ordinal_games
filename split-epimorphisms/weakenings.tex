\documentclass[11pt]{article} % use larger type; default would be 10pt

\usepackage[utf8]{inputenc}

\usepackage{graphicx} % support the \includegraphics command and options

\usepackage[parfill]{parskip} % Activate to begin paragraphs with an empty line rather than an indent

%%% PACKAGES
\usepackage{booktabs} % for much better looking tables
\usepackage{array} % for better arrays (eg matrices) in maths
\usepackage{paralist} % very flexible & customisable lists (eg. enumerate/itemize, etc.)
\usepackage{verbatim} % adds environment for commenting out blocks of text & for better verbatim
\usepackage{subfig} % make it possible to include more than one captioned figure/table in a single float
\usepackage{mathtools} % for the all important \coloneqq symbol
\usepackage{url} % for \url
\usepackage{IEEEtrantools} % for \IEEEeqnarray

%%% ToC (table of contents) APPEARANCE
\usepackage[nottoc,notlof,notlot]{tocbibind} % Put the bibliography in the ToC
\usepackage[titles,subfigure]{tocloft} % Alter the style of the Table of Contents
\renewcommand{\cftsecfont}{\rmfamily\mdseries\upshape}
\renewcommand{\cftsecpagefont}{\rmfamily\mdseries\upshape} % No bold!

%% Font things %%
\usepackage{amsmath}
\usepackage{amsfonts}
\usepackage{amssymb}
\usepackage{stmaryrd}
\usepackage{mathtools}
\usepackage{cmll} % Linear logic symbols!

%% Lists %%
\usepackage{enumerate}

%% Graphics %%
\usepackage{tikz}
\usetikzlibrary{cd}
\usetikzlibrary{patterns}
\usetikzlibrary{calc}

%% Theorems! %%
\usepackage{amsthm}
\theoremstyle{plain} % Theorems, lemmas, propositions etc.
\newtheorem{theorem}{Theorem}[section]
\newtheorem{lemma}[theorem]{Lemma}
\newtheorem{proposition}[theorem]{Proposition}
\newtheorem{corollary}[theorem]{Corollary}
\newtheorem{fact}[theorem]{Fact}
\newtheorem{construction}[theorem]{Construction}
\theoremstyle{definition} % Definitions etc.  Remarks too, because I don't like the way the 'remark' style looks.
\newtheorem{definition}[theorem]{Definition}
\newtheorem{notation}[theorem]{Notation}
\newtheorem{remark}[theorem]{Remark}
\newtheorem{example}[theorem]{Example}
\newtheorem{question}[theorem]{Question}

%% Exercises and answers %%
\usepackage{answers}

\newtheoremstyle{exercisestyle}% name
  {6pt}   % ABOVESPACE
  {6pt}   % BELOWSPACE
  {\itshape}  % BODYFONT
  {0pt}       % INDENT (empty value is the same as 0pt)
  {\bfseries} % HEADFONT
  {.}         % HEADPUNCT
  {3pt} % HEADSPACE
  {}          % CUSTOM-HEAD-SPEC

\theoremstyle{exercisestyle}
\newtheorem{exercise}{Exercise}
\newtheorem{answerthm}{Exercise}

\Newassociation{answer}{answerthm}{answers}
\newcommand{\answerthmparams}{}

%% Changes to enumerate things so they look better %%

\makeatletter
\def\enumfix{%
\if@inlabel
 \noindent \par\nobreak\vskip-\topsep\hrule\@height\z@
\fi}

\let\olditemize\itemize
\def\itemize{\enumfix\olditemize}
\let\oldenumerate\enumerate
\def\enumerate{\enumfix\oldenumerate}

%% Random crap %%
\usepackage{xifthen}

\makeatletter
\def\thm@space@setup{%
  \thm@preskip=\parskip \thm@postskip=0pt
}
\makeatother

\makeatletter
\newcommand*{\relrelbarsep}{.386ex}
\newcommand*{\relrelbar}{%
  \mathrel{%
    \mathpalette\@relrelbar\relrelbarsep
  }%
}
\newcommand*{\@relrelbar}[2]{%
  \raise#2\hbox to 0pt{$\m@th#1\relbar$\hss}%
  \lower#2\hbox{$\m@th#1\relbar$}%
}
\providecommand*{\rightrightarrowsfill@}{%
  \arrowfill@\relrelbar\relrelbar\rightrightarrows
}
\providecommand*{\leftleftarrowsfill@}{%
  \arrowfill@\leftleftarrows\relrelbar\relrelbar
}
\providecommand*{\xrightrightarrows}[2][]{%
  \ext@arrow 0359\rightrightarrowsfill@{#1}{#2}%
}
\providecommand*{\xleftleftarrows}[2][]{%
  \ext@arrow 3095\leftleftarrowsfill@{#1}{#2}%
}
\makeatother

\newcommand{\catname}[1]{{\normalfont\textbf{#1}}}
\newcommand{\Rings}{\catname{CRing}}
\newcommand{\CAT}{\catname{CAT}}
\newcommand{\Top}{\catname{Top}}
\newcommand{\Set}{\catname{Set}}
\newcommand{\Cont}{\catname{Cont}}
\newcommand{\Sch}{\catname{Sch}}
\newcommand{\Rel}{\catname{Rel}}
\newcommand{\Mod}[1][]{\ifthenelse{\isempty{#1}}{\catname{Mod}}{#1\catname{mod}}}
\DeclareMathOperator{\sh}{Sh}
\newcommand{\Sh}[1][]{\ifthenelse{\isempty{#1}}{\sh}{\sh(#1)}}
\newcommand{\map}[3]{#2\xrightarrow{#1} #3}
\newcommand*\from{\colon}
\newcommand{\cmap}[3]{#1\from{}#2\to{}#3}
\newcommand\oppcat[1]{#1^{\mathrm{op}}}
\DeclareRobustCommand{\vmap}[3] {\begin{tikzcd} #2 \arrow[d, "#1"] \\ #3 \end{tikzcd}}
\newcommand{\partref}[1]{(\ref{#1})}
\newcommand{\intgrpd}[4] {#1 \xrightrightarrows[#3]{#4} #2}
\DeclareRobustCommand{\bigintgrpd}[4] {\begin{tikzcd}[ampersand replacement=\&] #1 \arrow[r, shift left=0.5ex, "#3"] \arrow[r, shift right=0.5ex, "#4"'] \& #2 \end{tikzcd}}

\usepackage{xspace}

\newcommand{\etale}{\'{e}tale\xspace}
\newcommand{\Etale}{\'{E}tale\xspace}

\def \inv {^{-1}}

\DeclareMathOperator{\id}{id}
\DeclareMathOperator{\op}{op}
\DeclareMathOperator{\pr}{pr}
\DeclareMathOperator{\pre}{{pre}}
\DeclareMathOperator{\et}{{\acute{e}t}}

\DeclareMathOperator{\Hom}{Hom}
\DeclareMathOperator{\Spec}{Spec}

\DeclareMathOperator{\ol}{ol}

\def\presuper#1#2%
  {\mathop{}%
   \mathopen{\vphantom{#2}}^{#1}%
   \kern-\scriptspace%
   #2}
\def\presub#1#2%
  {\mathop{}%
   \mathopen{\vphantom{#2}}_{#1}%
   \kern-\scriptspace%
   #2}

%% Our things %%

\newcommand{\neggame}[1]{\presuper{\perp}{#1}}
\newcommand{\tensor}{\otimes}
\newcommand{\sequoid}{\oslash}
\newcommand{\varsequoid}{\vartriangleleft}
\renewcommand{\implies}{\multimap}
\newcommand{\comp}[2]{#1 \circ #2}
\newcommand{\cprd}{\sqcup}
\newcommand{\G}{\mathcal G}
\newcommand{\suchthat}{\;\colon\;}
\newcommand{\varsuchthat}{\;\mid\;}
\newcommand{\esuchthat}{\;.\;}
\newcommand{\OP}{\{O,P\}}
\newcommand{\F}{\mathcal F}
\renewcommand{\L}{\mathcal L}
\DeclareMathOperator{\wk}{wk}
\DeclareMathOperator{\weak}{weak}
\DeclareMathOperator{\str}{str}
\newcommand{\s}{\mathfrak{s}}
\renewcommand{\t}{\mathfrak{t}}
\newcommand{\grel}[1]{\underline{#1}}

%%% END Article customizations

\begin{document}

\section{The Hyland-Schalke functor $\G\to\Rel$}
\label{hylandschalke}

Let $\G$ denote an appropriate category of games and strategies, and let $\Rel$ denote the category of sets and relations.  It is known (Hyland, Schalke, 1996) that the following define a functor $\cmap{\F}{\G}{\Rel}$:

\begin{IEEEeqnarray*}{CCC}
  \F(A) & = & P_A \\
  \F\left(\map{\sigma}{A}{B}\right) & = & \{(\s\vert_A, \s\vert_B)\suchthat \s\in\sigma\textrm{ is a $P$-play}\}
\end{IEEEeqnarray*}

It can be shown that this functor $\F$ is faithful (see Calderon and McCusker, 2010 for a proof).  This surprising fact will be very useful for us: if we want to show that a particular diagram commutes in $\G$, we can normally do so more easily by applying the functor $\F$ to the diagram and showing that the resulting diagram commutes in $\Rel$.

\begin{notation}
  We shall simplify our notation by writing $P_A$ rather than $\F(A)$ and writing $\grel{\sigma}$ rather than $\F(\sigma)$.
\end{notation}

\section{Weakenings}

Let $A=(M_A,\lambda_A,P_A)$ be a game.  Then we have two notions of a \emph{weakening} of $A$ - informally, a game that is \emph{easier} than $A$, from the player $P$'s point of view.  

\begin{enumerate}
  \item An \emph{antecedent weakening} of $A$ is a game $A'=(M_A,\lambda_A,P_{A'})$, where $P_{A'}\subset P_{A}$ such that for every $O$-play $s$ occuring in both $P_A$ and $P_{A'}$, the set of $P$-responses to $s$ in $A'$ is precisely the set of $P$-responses to $s$ in $A$ and for every $P$-play $t$ occurring in both $P_A$ and $P_{A'}$, the set of $O$-responses to $t$ in $A'$ is a subset of the set of $O$-responses to $t$ in $A$.  
  \item A \emph{precedent strengthening} of $A$ is a game $A''=(M_A,\lambda_A,P_{A''})$ where $P_A\subset P_{A''}$ such that for every $O$-play $s$ occuring in both $P_A$ and $P_{A''}$, the set of player responses to $s$ in $A$ is a subset of the set of $P$-responses to $s$ in $A''$, and such that for every $P$-play $t$ occuring in both $P_A$ and $P_{A''}$ the set of $O$-responses to $t$ in $A$ is precisely the set of $O$-responses to $t$ in $A''$.  
\end{enumerate}

In other words, a precedent strengthening of $A$ is a game in which the player has more choices at each move, but the opponent has the same choices, while an antecedent weakening of $A$ is a game in which the player has the same choices at each move, but the opponent has fewer choices.  

We can express the definition of an antecedent weakening more compactly as follows: $(M_A, \lambda_A, P_{A'})$ is an antecedent weakening of $(M_A, \lambda_A, P_A)$ if $P_{A'}\subset P_A$ and the following condition holds: whenever $s\in P_{A'}$ is an $O$-play and $a\in M_A$ is a $P$-move such that $sa\in P_A$, then $sa\in P_{A'}$.  This condition says that player $P$ has all the moves available to her in $A'$ that she had in $P_A$ - the other conditions mentioned above follow automatically from the fact that $P_{A'}\subset P_A$.

The important fact about both types of weakening is that we have copycat morphisms $A\to A',A\to A''$.  

From here on, we will use the term \emph{weakening} to refer to an antecedent weakening.  If $A'$ is a weakening of $A$, we will write $\cmap{\wk}{A}{A'}$ for the copycat morphism.  

\begin{proposition}
  Let $A'$ be a weakening of $A$.  Then $\cmap{\wk}{A}{A'}$ has a right inverse.  

  \begin{proof}
    The inverse for $\wk$ is the partial strategy $\cmap{\str}{A'}{A}$.  This is also constructed as a copycat strategy; the only difference is that if $O$ makes a move in $A$ that does not occur in $P_{A'}$ then the player $P$ has no reply.  We now claim that $\comp\wk\str=\id_{A'}$.  

    Formally:
    \begin{IEEEeqnarray*}{rCl}
      \wk & = & \left\{s\in P_{A\implies A'}\suchthat\textrm{if $|s|$ is even, then $s\vert_A=s\vert_{A'}$}\right\} \\
      \str & = & \left\{s\in P_{A'\implies A}\suchthat\textrm{if $|s|$ is even, then $s\vert_A'=s\vert_{A}$}\right\}
    \end{IEEEeqnarray*}

    We wish to show that $\comp\wk\str=\id_{A'}$.  In the style of section \ref{hylandschalke}, we shall show that $\comp{\grel\wk}{\grel\str}=\Delta_{P_{A'}}$ in $\Rel$.

    It is easy to see that
    \begin{IEEEeqnarray*}{rCl}
      \grel\wk & = & \{(s, s)\suchthat s\in P_{A'}\}\subset P_A\times P_{A'} \\
      \grel\str & = & \{(s, s)\suchthat s\in P_{A'}\}\subset P_{A'}\times P_A
    \end{IEEEeqnarray*}

    It is then easy to see that $\comp\wk\str=\Delta_{A'}\subset P_{A'}\times P_{A'}$, as we wanted.  As a sanity check, if we try to take the composition the other way round then we get $\Delta_{A'}\subset P_A\times P_A$, which is not the identity unless $P_{A'}=P_A$.
  \end{proof}
\end{proposition}

\section{The weakening monad of a game}

\subsection{Weakening pullback of a game}

Let $A$ be a game.  Suppose $B$ is a game and $\cmap{\sigma}{B}{A}$ is a morphism.  Define
\[
  P_{B'}=\{\s\vert_B\suchthat \s\in\sigma\}
  \]
$P_{B'}$ is prefix closed, since $\sigma$ is, so $B'=(M_B, \lambda_B, P_{B'})$ is a well-formed game.  We claim that $B'$ is a weakening of $B$.  Indeed, suppose $s\in P_{B'}$ is an $O$-play and $sa\in P_B$.  By the definition of $P_{B'}$, there must be some $\s\in\sigma$ such that $\s\vert_B=s$; choosing $\s$ minimal, we may assume that the last move in $\s$ coincides with the last move of $s$.  Since $s$ is an $O$-play, this last move is an $O$-move in $B$ and therefore a $P$-move in $B\implies A$.  Therefore, $\s a\in \sigma$ since $\sigma$ is a strategy and so $sa=\s a\vert_B\in P_{B'}$.  We call $B'$ the \emph{weakening pullback} of $A$ along the morphism $\sigma$.

Next, we define a morphism $\cmap{\sigma'}{B'}{A}$.  As sets, $\sigma'=\sigma$ (noting that, by definition, $\sigma\subset P_{B'\implies A}$).  We claim that
\[
  \begin{tikzcd}
    B \arrow[r, "\wk"] \arrow[dr, "\sigma"']
      & B' \arrow[d, "\sigma'"] \\
    %
      & A
  \end{tikzcd}
  \]
commutes.

Once again, we check this in $\Rel$.  If we note that $\grel\wk=\Delta_{P_{B'}}\subset P_B\times P_{B'}$ and that $\grel\sigma=\grel{\sigma'}$ as sets, then it is clear that $\comp{\grel{\sigma'}}{\grel\wk}=\grel\sigma$ in $\Rel$, and so $\comp{\sigma'}{\wk}=\sigma$ in $\G$.

The weakening $B'$ and the morphism $\sigma'$ satisfy a universal property - namely, if $B''$ is a weakening of $B$ and $\cmap{\sigma''}{B''}{A}$ is a morphism such that
\begin{equation}
  \label{pullback.b.double.prime}
  \begin{tikzcd}
    B \arrow[r, "\wk"] \arrow[dr, "\sigma"']
      & B'' \arrow[d, "\sigma''"] \\
    %
      & A
  \end{tikzcd}
\end{equation}
commutes, then $B'$ is a weakening of $B''$ and 
\begin{equation}
  \label{pullback.b.prime}
  \begin{tikzcd}
    B'' \arrow[r, "\wk"] \arrow[dr, "\sigma''"']
      & B' \arrow[d, "\sigma'"] \\
    %
      & A
  \end{tikzcd}
\end{equation}
commutes.  Indeed, working in $\Rel$ again, diagram \eqref{pullback.b.double.prime} tells us that $\grel\sigma=\grel{\sigma''}$ as sets, which implies that $\sigma\subset P_{B''}\times P_A$.  Noting the definition of $P_{B'}$, it follows that $\map{\sigma'}{B'}{A}$ is the weakening pullback of $A$ along $\sigma''$, and so diagram \eqref{pullback.b.prime} commutes.

\subsection{The weakening monad}

We shall want to prove various useful properties of these weakening pullbacks, notably a functoriality condition: if $C\xrightarrow{\tau}\map{\sigma}{B}{A}$ are morphisms, then we get a morphism $\cmap{\tau'}{C'}{B'}$ such that $(\comp\sigma\tau)'=\comp{\sigma'}{\tau'}$.  This and other useful facts are brought together in the following theorem:

\begin{theorem}
  Let $A$ be a game.  Then we have a functor $\cmap{\weak_A}{\G/A}{\G/A}$ that sends a morphism $\map{\sigma}{B}{A}$ to the pullback weakening $\map{\sigma'}{B'}{A}$ and that gives rise to an idempotent monad on $\G/A$.

  \begin{proof}
    In order to complete the definition of the functor $\weak_A$, we need to define its action on morphisms.  Let $C\xrightarrow{\tau}\map{\sigma}{B}{A}$ be morphisms in $\G$ (so $\tau$ gives us a morphism from $\map{\comp\sigma\tau}{C}{A}$ to $\map{\sigma}{B}{A}$ in $\G/A$).  We define a morphism $\cmap{\tau'}{C'}{B'}$ by $\tau'=\tau\cap P_{C'\implies B'}$.

    It is easy to see that $\tau'$ is a strategy, so we need to check that $\comp{\sigma'}{\tau'}=(\comp\sigma\tau)'$.  As usual, we perform this check in $\Rel$.

    Since $(\comp\sigma\tau)'$ is created according to the pullback recipe given above, we have $\grel{(\comp\sigma\tau)'}=\grel{\comp\sigma\tau}$ as sets.  So, as subsets of $P_{C'}\times P_A$, we have:
    \begin{align*}
      \grel{(\comp\sigma\tau)'} & = \grel{\comp\sigma\tau} \\
                                & = \comp{\grel\sigma}{\grel\tau} \\
                                & = \{(s, t)\in P_{C'}\times P_{A}\suchthat \exists u\in P_B \esuchthat (s,u)\in\grel\tau, (u,t)\in\grel\sigma\}
    \end{align*}

    Meanwhile, as sets we have $\grel{\sigma'}=\grel\sigma$ and $\grel{\tau'}=\grel\tau\cap (P_{C'}\times P_{B'})$.  So, as subsets of $P_{C'}\times P_A$, we have:
    \begin{align*}
      \comp{\grel{\sigma'}}{\grel{\tau'}} & = \{(s, t)\in P_{C'}\times P_A\suchthat \exists u\in P_{B'} \esuchthat (s,u)\in\grel\tau, (u,t)\in\grel\sigma\}
    \end{align*}

     - in other words, the same thing, but with $P_{B'}$ instead of $P_B$.  It is clear then that $\comp{\grel{\sigma'}}{\grel{\tau'}}\subset\grel{(\comp\sigma\tau)'}$.  In the other direction, suppose $(s,t)\in P_{C'}\times P_A$ and that there exiss $u\in P_B$ such that $(s,u)\in\grel\tau$ and $(u,t)\in\grel\sigma$.  But now, since $(u,t)\in\grel\sigma$, we must have $u\in P_{B'}$ by the definition of $P_{B'}$ and therefore $(u,t)\in \comp{\grel{\sigma'}}{\grel{\tau'}}$.

     So far what we have done is show that the morphism $\tau'$ does indeed give us a morphism in $\G/A$ from $\map{(\comp\sigma\tau)'}{C'}{A}$ to $\map{\sigma'}{B'}{A}$.  We still need to prove that what we have defined respects composition and so gives us a functor $\G/A\to \G/A$.

     This means proving the following statement: suppose $\map{\upsilon}{D}{C}\xrightarrow{\tau}\map{\sigma}{B}{A}$ are morphisms in $\G$.  This gives us morphisms $\upsilon,\tau$ in $\G/A$ as pictured in the diagram below:
    \[
      \begin{tikzcd}
        D \arrow[r, "\upsilon"] \arrow[drr, "\comp{\sigma}{\comp{\tau}{\upsilon}}"']
          & C \arrow[r, "\tau"] \arrow[dr, "\comp\sigma\tau"]
            & B \arrow[d, "\sigma"] \\
        %
          &
            & A
      \end{tikzcd}
      \]
    We have $\cmap{\upsilon'}{D'}{C'}$ given by $\upsilon'=\upsilon\cap P_{D'\implies C'}$ and $\cmap{\tau'}{C'}{B'}$ given by $\tau'=\tau\cap P_{C'\implies B'}$, as before.  We also have a morphism $\cmap{(\comp\tau\upsilon)'}{D'}{B'}$ given by $(\comp\tau\upsilon)'=(\comp\tau\upsilon)\cap P_{D'\implies B'}$.  We need to show that $(\comp\tau\upsilon)'=\comp{\tau'}{\upsilon'}$.

    We have $(\comp\tau\upsilon)'=(\comp\tau\upsilon)\cap P_{D'\implies B'}$, so:
    \begin{align*}
      \grel{(\comp\tau\upsilon)'} & = \grel{(\comp\tau\upsilon)}\cap(P_{D'}\times P_{B'}) \\
       & = (\comp{\grel\tau}{\grel\upsilon}) \cap (P_{D'}\times P_{B'}) \\
       & = \{(s,t)\in P_{D'}\times P_{B'} \suchthat \exists u\in P_C \esuchthat (s,u) \in \grel\upsilon\;, (u,t) \in \grel\tau \}
    \end{align*}

    Meanwhile, $\grel{\tau '} = \grel\tau \cap P_{D'}$ and $\grel{\upsilon '} = \grel\upsilon \cap P_{B'}$, so
    \[
      \comp{\grel{\tau '}}{\grel{\upsilon '}} = \{(s,t) \in P_{D'}\times P_{B'} \suchthat \exists u\in P_{C'} \esuchthat (s,u) \in \grel\upsilon\; (u,t)\in \grel\tau \}
      \]
    which is the same thing, but with $P_{C'}$ rather than $P_C$.  

    Clearly, $\comp{\grel{\tau '}}{\grel{\upsilon '}} \subseteq \grel{(\comp\tau\upsilon)'}$.  In the other direction, suppose that $s\in P_{D'}, t\in P_{B'}$ and that $u\in P_C$ such that $(s,u)\in \grel\upsilon,(u,t)\in\grel\tau$.  It will suffice to show that $u\in P_{C'}$.  Now note that $C'$ is pullback of $A$ along $\comp\sigma\tau$ and that $B'$ is pullback of $A$ along $\sigma$.  Therefore:
    \begin{align*}
      P_{C'} & = \{\s\vert_C\suchthat \s\in\comp\sigma\tau\} \\
      P_{B'} & = \{\t\vert_B\suchthat \t\in\sigma\}
    \end{align*}
    Since $t\in P_{B'}$, there must be some $\t\in \sigma$ such that $\t\vert_B=t$.  Put another way, there must be some $v\in P_A$ such that $(t,v)\in\grel\sigma$.  We have $(u,t)\in\grel\tau$, so therefore $(u,v)\in\grel{\comp\sigma\tau}$, and it follows that $u\in P_{C'}$.  

    Therefore, we have that $\grel{\comp{\tau'}{\upsilon'}}=\comp{\grel{\tau '}}{\grel{\upsilon '}} = \grel{(\comp\tau\upsilon)'}$.  Since the Hyland-Schalke functor is faithful, this means that $\comp{\tau '}{\upsilon '}=(\comp\tau\upsilon)'$.

    This all means that we can define a functor $\cmap{\weak_A}{\G/A}{\G/A}$ by
    \begin{IEEEeqnarray*}{CCCCC}
      \map{\sigma}{B}{A} & \quad & \mapsto & \quad & \map{\sigma'}{B'}{A} \\
      \begin{tikzcd}
        C \arrow[r, "\tau"] \arrow[dr, "\comp\sigma\tau"']
         & B \arrow[d, "\sigma"] \\
        %
         & A
      \end{tikzcd}
      & \quad & \mapsto & \quad &
      \begin{tikzcd}
        C' \arrow[r, "\tau'"] \arrow[dr, "\comp{\sigma'}{\tau'}"']
         & B' \arrow[d, "\sigma'"] \\
        %
         & A
      \end{tikzcd}
    \end{IEEEeqnarray*}

    We now define the monad structure on $\G/A$.  The unit transformation $\cmap{\eta_A}{\id_{\G/A}}{\weak_A}$ is given by the weakening morphisms $\cmap{\wk}{B}{B'}$ for each object $\map{\sigma}{B}{A}$; we have already shown that the diagrams
    \[
      \begin{tikzcd}
        B \arrow[r, "\wk"] \arrow[dr, "\sigma"']
         & B' \arrow[d, "\sigma'"] \\
        %
         & A
      \end{tikzcd}
      \]
    commute.  To show that this is indeed a natural transformation, we need to show that if we are given some morphism in $\G/A$
    \[
      \begin{tikzcd}
        C \arrow[r, "\tau"] \arrow[dr, "\comp\sigma\tau"']
          & B \arrow[d, "\sigma"] \\
        %
          & A
      \end{tikzcd}
      \]
    then the following square commutes:
    \[
      \begin{tikzcd}
        C \arrow[r, "\wk"] \arrow[d, "\tau"']
          & C' \arrow[d, "\tau'"] \\
        B \arrow[r, "\wk"']
          & B'
      \end{tikzcd}
      \]
    To show this, we work in $\Rel$ again.  On the one hand, we have:
    \begin{align*}
      \comp{\grel{\tau'}}{\grel{\wk}} & = \{(s,t)\in P_C\times P_{B'}\suchthat\exists u\in P_{C'}\esuchthat (s,u)\in\grel\wk\;,(u,t)\in\grel{\tau'}\} \\
       & = \{(s, t)\in P_{C'}\times P_{B'}\suchthat (s,t)\in\grel{\tau'}\} \\
       & = \grel{\tau'} = \grel{\tau}\cap(P_{C'}\times P_{B'})
    \end{align*}
    On the other hand:
    \begin{align*}
      \comp{\grel\wk}{\grel\tau} & = \{(s, t)\in P_C\times P_{B'}\suchthat \exists u\in P_B\esuchthat (s,u)\in \grel\tau, (u, t)\in \grel\wk\} \\
       & = \grel\tau \cap (P_C\times P_{B'})
    \end{align*}
    
    We certainly have $\comp{\grel{\tau'}}{\grel\wk}\subseteq\comp{\grel\wk}{\grel\tau}$.  In the other direction, suppose that $s\in P_C$ and $t\in P_{B'}$ such that $(s,t)\in\grel\tau$.  It will suffice to show that $s\in P_{C'}$.  Indeed, by the definition of $P_{B'}$, there is some $v\in P_A$ such that $(t, v)\in\grel\sigma$.  Therefore, $(s, v)\in\grel{\comp{\sigma}{\tau}}$, and so $s\in P_{C'}$, as required.  

   Before defining the multiplication $\cmap{\mu_A}{\comp{\weak_A}{\weak_A}}{\weak_A}$ in the monad, we look at the identities that this unit transformation $\eta_A$ will have to satisfy.  If $\map{\sigma}{B}{A}$ is an object of $\G/A$ then we want the diagrams
   \[
     \begin{tikzcd}
       B' \arrow[r, "\wk"] \arrow[dr, equals]
         & B'' \arrow[d, "\mu_A(B)"] \\
       %
         & B'
     \end{tikzcd}
     \textrm{ and }
     \begin{tikzcd}
       B' \arrow[r, "\wk'"] \arrow[dr, equals]
         & B'' \arrow[d, "\mu_A(B)"] \\
       %
         & B'
     \end{tikzcd}
     \]
   to commute.  The arrow $\cmap{\wk}{B'}{B''}$ in the first diagram is the weakening arrow obtained from the object $\map{\sigma'}{B'}{A}$ of $\G/A$ using the natural transformation $\eta_A$, while the arrow $\cmap{\wk'}{B'}{B''}$ in the second diagram is obtained by applying the functor $\weak_A$ to the morphism $\cmap{\wk}{B}{B'}$.  

   In practice, however, there is no need for this distinction: $B''$ is the same game as $B'$ and the arrows $\wk$ and $\wk'$ are the identity.  To see this, recall the definitions of $B'$ and $\sigma'$:
   \begin{align*}
     P_{B'} & = \{\s\vert_B\suchthat \s\in\sigma\} \\
     \sigma' & = \sigma\textrm{ (as sets of plays)}
   \end{align*}
   Then we should have $P_{B''} = \{\s\vert_{B'}\suchthat \s\in\sigma'\}$, but since $\sigma'$ is the same set as $\sigma$, this is exactly the same as the definition of $P_{B'}$.  

   Since $\wk,\wk'$ are identity maps, we observe that we have the strict equality $\comp{\weak_A}{\weak_A}=\weak_A$ and we had better define $\mu_A$ to be the identity transformation.  This $\mu_A$ is automatically associative and the unit diagrams above hold (since all maps involved are identity maps).  We end up with an idempotent monad on $\G/A$.
  \end{proof}
\end{theorem}

\end{document}
