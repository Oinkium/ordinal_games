\documentclass[11pt]{article} % use larger type; default would be 10pt

\usepackage[utf8]{inputenc}

\usepackage{graphicx} % support the \includegraphics command and options

\usepackage[parfill]{parskip} % Activate to begin paragraphs with an empty line rather than an indent

%%% PACKAGES
\usepackage{booktabs} % for much better looking tables
\usepackage{array} % for better arrays (eg matrices) in maths
\usepackage{paralist} % very flexible & customisable lists (eg. enumerate/itemize, etc.)
\usepackage{verbatim} % adds environment for commenting out blocks of text & for better verbatim
\usepackage{subfig} % make it possible to include more than one captioned figure/table in a single float
\usepackage{mathtools} % for the all important \coloneqq symbol
\usepackage{url} % for \url
\usepackage{IEEEtrantools} % for \IEEEeqnarray

%%% ToC (table of contents) APPEARANCE
\usepackage[nottoc,notlof,notlot]{tocbibind} % Put the bibliography in the ToC
\usepackage[titles,subfigure]{tocloft} % Alter the style of the Table of Contents
\renewcommand{\cftsecfont}{\rmfamily\mdseries\upshape}
\renewcommand{\cftsecpagefont}{\rmfamily\mdseries\upshape} % No bold!

%% Font things %%
\usepackage{amsmath}
\usepackage{amsfonts}
\usepackage{amssymb}
\usepackage{stmaryrd}
\usepackage{mathtools}
\usepackage{cmll} % Linear logic symbols!

%% Lists %%
\usepackage{enumerate}

%% Graphics %%
\usepackage{tikz}
\usetikzlibrary{cd}
\usetikzlibrary{patterns}
\usetikzlibrary{calc}

%% Theorems! %%
\usepackage{amsthm}
\theoremstyle{plain} % Theorems, lemmas, propositions etc.
\newtheorem{theorem}{Theorem}[section]
\newtheorem{lemma}[theorem]{Lemma}
\newtheorem{proposition}[theorem]{Proposition}
\newtheorem{corollary}[theorem]{Corollary}
\newtheorem{fact}[theorem]{Fact}
\newtheorem{construction}[theorem]{Construction}
\theoremstyle{definition} % Definitions etc.  Remarks too, because I don't like the way the 'remark' style looks.
\newtheorem{definition}[theorem]{Definition}
\newtheorem{notation}[theorem]{Notation}
\newtheorem{remark}[theorem]{Remark}
\newtheorem{example}[theorem]{Example}
\newtheorem{question}[theorem]{Question}

%% Exercises and answers %%
\usepackage{answers}

\newtheoremstyle{exercisestyle}% name
  {6pt}   % ABOVESPACE
  {6pt}   % BELOWSPACE
  {\itshape}  % BODYFONT
  {0pt}       % INDENT (empty value is the same as 0pt)
  {\bfseries} % HEADFONT
  {.}         % HEADPUNCT
  {3pt} % HEADSPACE
  {}          % CUSTOM-HEAD-SPEC

\theoremstyle{exercisestyle}
\newtheorem{exercise}{Exercise}
\newtheorem{answerthm}{Exercise}

\Newassociation{answer}{answerthm}{answers}
\newcommand{\answerthmparams}{}

%% Changes to enumerate things so they look better %%

\makeatletter
\def\enumfix{%
\if@inlabel
 \noindent \par\nobreak\vskip-\topsep\hrule\@height\z@
\fi}

\let\olditemize\itemize
\def\itemize{\enumfix\olditemize}
\let\oldenumerate\enumerate
\def\enumerate{\enumfix\oldenumerate}

%% Random crap %%
\usepackage{xifthen}

\makeatletter
\def\thm@space@setup{%
  \thm@preskip=\parskip \thm@postskip=0pt
}
\makeatother

\makeatletter
\newcommand*{\relrelbarsep}{.386ex}
\newcommand*{\relrelbar}{%
  \mathrel{%
    \mathpalette\@relrelbar\relrelbarsep
  }%
}
\newcommand*{\@relrelbar}[2]{%
  \raise#2\hbox to 0pt{$\m@th#1\relbar$\hss}%
  \lower#2\hbox{$\m@th#1\relbar$}%
}
\providecommand*{\rightrightarrowsfill@}{%
  \arrowfill@\relrelbar\relrelbar\rightrightarrows
}
\providecommand*{\leftleftarrowsfill@}{%
  \arrowfill@\leftleftarrows\relrelbar\relrelbar
}
\providecommand*{\xrightrightarrows}[2][]{%
  \ext@arrow 0359\rightrightarrowsfill@{#1}{#2}%
}
\providecommand*{\xleftleftarrows}[2][]{%
  \ext@arrow 3095\leftleftarrowsfill@{#1}{#2}%
}
\makeatother

\newcommand{\catname}[1]{{\normalfont\textbf{#1}}}
\newcommand{\Rings}{\catname{CRing}}
\newcommand{\CAT}{\catname{CAT}}
\newcommand{\Top}{\catname{Top}}
\newcommand{\Set}{\catname{Set}}
\newcommand{\Cont}{\catname{Cont}}
\newcommand{\Sch}{\catname{Sch}}
\newcommand{\Rel}{\catname{Rel}}
\newcommand{\Mod}[1][]{\ifthenelse{\isempty{#1}}{\catname{Mod}}{#1\catname{mod}}}
\DeclareMathOperator{\sh}{Sh}
\newcommand{\Sh}[1][]{\ifthenelse{\isempty{#1}}{\sh}{\sh(#1)}}
\newcommand{\map}[3]{#2\xrightarrow{#1} #3}
\newcommand*\from{\colon}
\newcommand{\cmap}[3]{#1\from{}#2\to{}#3}
\newcommand\oppcat[1]{#1^{\mathrm{op}}}
\DeclareRobustCommand{\vmap}[3] {\begin{tikzcd} #2 \arrow[d, "#1"] \\ #3 \end{tikzcd}}
\newcommand{\partref}[1]{(\ref{#1})}
\newcommand{\intgrpd}[4] {#1 \xrightrightarrows[#3]{#4} #2}
\DeclareRobustCommand{\bigintgrpd}[4] {\begin{tikzcd}[ampersand replacement=\&] #1 \arrow[r, shift left=0.5ex, "#3"] \arrow[r, shift right=0.5ex, "#4"'] \& #2 \end{tikzcd}}

\usepackage{xspace}

\newcommand{\etale}{\'{e}tale\xspace}
\newcommand{\Etale}{\'{E}tale\xspace}

\def \inv {^{-1}}

\DeclareMathOperator{\id}{id}
\DeclareMathOperator{\op}{op}
\DeclareMathOperator{\pr}{pr}
\DeclareMathOperator{\pre}{{pre}}
\DeclareMathOperator{\et}{{\acute{e}t}}

\DeclareMathOperator{\Hom}{Hom}
\DeclareMathOperator{\Spec}{Spec}

\DeclareMathOperator{\ol}{ol}

\def\presuper#1#2%
  {\mathop{}%
   \mathopen{\vphantom{#2}}^{#1}%
   \kern-\scriptspace%
   #2}
\def\presub#1#2%
  {\mathop{}%
   \mathopen{\vphantom{#2}}_{#1}%
   \kern-\scriptspace%
   #2}

%% Our things %%

\newcommand{\neggame}[1]{\presuper{\perp}{#1}}
\newcommand{\tensor}{\otimes}
\newcommand{\sequoid}{\oslash}
\newcommand{\varsequoid}{\vartriangleleft}
\renewcommand{\implies}{\multimap}
\newcommand{\comp}[2]{#1 \circ #2}
\newcommand{\cprd}{\sqcup}
\newcommand{\G}{\mathcal G}
\newcommand{\suchthat}{\;\colon\;}
\newcommand{\varsuchthat}{\;\mid\;}
\newcommand{\OP}{\{O,P\}}
\newcommand{\F}{\mathcal F}
\renewcommand{\L}{\mathcal L}
\DeclareMathOperator{\wk}{wk}
\DeclareMathOperator{\str}{str}
\newcommand{\s}{\mathfrak{s}}
\renewcommand{\t}{\mathfrak{t}}

%%% END Article customizations

\begin{document}

\section{Weakenings}

Let $A=(M_A,\lambda_A,P_A)$ be a game.  Then we have two notions of a \emph{weakening} of $A$ - informally, a game that is \emph{easier} than $A$, from the player $P$'s point of view.  

\begin{enumerate}
  \item An \emph{antecedent weakening} of $A$ is a game $A'=(M_A,\lambda_A,P_{A'})$, where $P_{A'}\subset P_{A}$ such that for every $O$-play $s$ occuring in both $P_A$ and $P_{A'}$, the set of $P$-responses to $s$ in $A'$ is precisely the set of $P$-responses to $s$ in $A$ and for every $P$-play $t$ occurring in both $P_A$ and $P_{A'}$, the set of $O$-responses to $t$ in $A'$ is a subset of the set of $O$-responses to $t$ in $A$.  
  \item A \emph{precedent strengthening} of $A$ is a game $A''=(M_A,\lambda_A,P_{A''})$ where $P_A\subset P_{A''}$ such that for every $O$-play $s$ occuring in both $P_A$ and $P_{A''}$, the set of player responses to $s$ in $A$ is a subset of the set of $P$-responses to $s$ in $A''$, and such that for every $P$-play $t$ occuring in both $P_A$ and $P_{A''}$ the set of $O$-responses to $t$ in $A$ is precisely the set of $O$-responses to $t$ in $A''$.  
\end{enumerate}

In other words, a precedent strengthening of $A$ is a game in which the player has more choices at each move, but the opponent has the same choices, while an antecedent weakening of $A$ is a game in which the player has the same choices at each move, but the opponent has fewer choices.  

The important fact about both these types of weakening is that we have copycat morphisms $A\to A',A\to A''$.  

From here on, we will use the term \emph{weakening} to refer to an antecedent weakening.  If $A'$ is a weakening of $A$, we will write $\cmap{\wk}{A}{A'}$ for the copycat morphism.  

\begin{proposition}
  Let $A'$ be a weakening of $A$.  Then $\cmap{\wk}{A}{A'}$ has a right inverse.  

  \begin{proof}
    The inverse for $\wk$ is the partial strategy $\cmap{\str}{A'}{A}$.  This is also constructed as a copycat strategy; the only difference is that if $O$ makes a move in $A$ that does not occur in $P_{A'}$ then the player $P$ has no reply.  We now claim that $\comp\wk\str=\id_{A'}$.  

    Formally:
    \begin{IEEEeqnarray*}{rCl}
      \wk & = & \left\{s\in P_{A\implies A'}\suchthat\textrm{if $|s|$ is even, then $s\vert_A=s\vert_{A'}$}\right\} \\
      \str & = & \left\{s\in P_{A'\implies A}\suchthat\textrm{if $|s|$ is even, then $s\vert_A'=s\vert_{A}$}\right\}
    \end{IEEEeqnarray*}

    To show that $\comp\wk\str=\id_{A'}$, it suffices to show that player $P$'s response, under $\comp\wk\str$, to an $O$-move in the left-hand copy of $A'$ is to copy that move into the right-hand copy of $A'$ and that her response to an $O$-move in the right hand copy is to copy that move into the left hand copy.  This is fairly straightforward to show inductively, using the narrative definition of composition: if $O$ makes a move in the left-hand copy of $A'$, then Player $P$'s response under $\str$ is to copy that move across into $A$.  Then, by $\wk$, Player $P$ copies the move over into the right-hand copy of $A'$, completing her move.  

    To go in the other direction, if Player $O$ plays in the right hand copy of $A'$, then Player $P$'s response under $\wk$ is to copy that move into $A$.  Now, since that move came from $A'$ originally, she is able to copy it over into the left-hand copy of $A'$.  Therefore, Player $P$'s overall strategy is the copycat strategy on $A'$.
  \end{proof}

\end{proposition}


We now show that is in fact a characterization of weakenings, in the sense that every strategy with a right inverse is \emph{equivalent} to some weakening strategy.  

Explicitly: 

\begin{proposition}
  If a morphism $\cmap{\sigma}{A}{B}$ has a right inverse, then there is a weakening $A'$ of $A$ and an isomorphism $\cmap{\hat{\sigma}}{A'}{B}$ such that
    \[
    \begin{tikzcd}
      A \arrow[r, "\sigma"] \arrow[dr, "\wk"']
        & B \arrow[d, leftarrow, "\rotatebox{90}{\(\sim\)}"', "\hat{\sigma}"] \\
      %
        & A'
    \end{tikzcd}
    \]
    commutes.

    \begin{proof}
      Let $\cmap{\sigma}{A}{B}$ be a morphism of games and suppose that $\sigma$ has a right inverse $\tau$ - so $\comp\sigma\tau=\id_B$.  Define
      \[
        P_{A'}=\{\s\vert_A\suchthat \s\in\sigma\}
        \]
      $P_{A'}$ is clearly prefix-closed, so $A'=(M_A,\lambda_A,P_{A'})$ is a well-defined game.

      We claim that $A'$ is a weakening of $A$.  It will suffice to show that if $s\in P_{A'}$ is an $O$-play and $sa\in P_A$ is a $P$-reply, then $sa\in P_{A'}$ (the other conditions follow automatically from the fact that $P_{A'}\subset P_A$).  

      Let $s\in P_{A'}$ be an $O$-play.  By the definition of $P_{A'}$, there exists $\s\in \sigma$ such that $\s\vert_A=s$.  We may choose $\s$ minimal, so that the last move made in $\s$ corresponds to the last move played in $s$.  This last move was an $O$-move in $A$, so it is a $P$-move in $A\implies B$.  This means that $\s$ is a $P$-play, and so $\s a\in \sigma$, since $\tau$ is a strategy.  Therefore, $\s a\vert_A=sa$, and so $sa\in P_{A'}$.  

      We now define strategies $\cmap{\hat\sigma}{A'}{B}, \cmap{\hat\tau}{B}{A'}$, given as sets of plays by:
      \begin{align*}
        \hat\sigma & = \sigma \\
        \hat\tau & = \tau \cap P_{B\implies A'}
      \end{align*}
      $\hat\tau$ is clearly a valid strategy for $B\implies A'$, and $\hat\sigma$ is a valid strategy for $A'\implies B$ by the definition of $P_{A'}$.

      We claim that $\sigma$ and $\tau$ are inverses and that the diagram given above commutes.  

      Since $\comp\sigma\tau=\id_B$, to show that $\comp{\hat\sigma}{\hat\tau}=\id_B$, it suffices to show that $\comp\sigma\tau=\id_B=\comp{\hat\sigma}{\hat\tau}$.  We have
      \begin{align*}
        \comp\sigma\tau & = \{\s\vert_{B,B}\suchthat \s\in \L(B,A,B), \s\vert_{B,A}\in\tau,\s\vert_{A,B}\in\sigma\} \\
        \comp{\hat\sigma}{\hat\tau} & = \{\s\vert_{B,B}\suchthat \s\in \L(B,A',B), \s\vert_{B,A'}\in\hat\tau,\s\vert_{A',B}\in\hat\sigma\}
      \end{align*}

      Since $\hat\tau\subset\tau$ and $\hat\sigma=\sigma$ as sets of plays, it is clear that $\comp{\hat\sigma}{\hat\tau}\subset\comp\sigma\tau$.  Going in the other direction, suppose $\s\in\L(B,A',B)$ is such that $\s\vert_{B,A}\in\tau$ and $\s\vert_{A,B}\in\sigma$.  Then, by definition,
      \[
        \s\vert_A=\s\vert_{A,B}\vert_A\in P_{A'}
        \]
      and so $\s\in\L(B,A',B)$ and $\s\vert_{B,A}\in P_{B\implies A'}$, which implies that $\s\vert_{B,A}\in\hat\tau$.  So $\comp\sigma\tau\subset\comp{\hat\sigma}{\hat\tau}$, and therefore $\comp{\hat\sigma}{\hat\tau}=\comp\sigma\tau=\id_B$.  
    \end{proof}
\end{proposition}

\end{document}
